\section{Linear Equations}

\subsection{Introduction to Linear Systems}
\begin{definition}
    A \textbf{Linear System} is a collection of $m$ linear equations with $n$ unknowns.
    \begin{align*}
        \begin{cases}
            a_{1,1}x_1 + a_{1,2}x_2 + \cdots + a_{1, n} x_n = b_1 \\
            a_{2,1}x_1 + a_{2,2}x_2 + \cdots + a_{2, n} x_n = b_2 \\
            \vdots \\
            a_{m,1}x_1 + a_{m,2}x_2 + \cdots + a_{m, n} x_n = b_m 
        \end{cases}
    \end{align*}
    Where $a_{i,j}, b_i \in \F$.
\end{definition}

\begin{notation}
    Given the following arbitrary linear system,
    \begin{align*}
        \begin{cases}
            a_{1,1}x_1 + a_{1,2}x_2 + \cdots + a_{1, n} x_n = b_1 \\
            a_{2,1}x_1 + a_{2,2}x_2 + \cdots + a_{2, n} x_n = b_2 \\
            \vdots \\
            a_{m,1}x_1 + a_{m,2}x_2 + \cdots + a_{m, n} x_n = b_m 
        \end{cases}
    \end{align*}

    We can write this as a system of matrices.
    \begin{align*}
        \begin{pmatrix}
            a_{1,1} & \cdots & a_{1, n} \\
            \vdots & \ddots & \vdots \\
            a_{m, 1} & \cdots & a_{m, n}
        \end{pmatrix}
        \vect{x_1 \\ \vdots \\ x_n} &=
        \vect{b_1 \\ \vdots \\ b_m}
    \end{align*}

    We can further simplify writing such an equation through an \textbf{augmented matrix}.
    \begin{align*}
        \left(\begin{array}{ccc|c}
            a_{1,1} & \cdots & a_{1,n} & b_1 \\
            \vdots  & \ddots & \vdots  & \vdots \\
            a_{m,1} & \cdots & a_{m,n} & b_m
        \end{array}\right)
    \end{align*}

    Also denoted as $(A | b)$.
\end{notation}

\subsection{Finding Solutions: Gauss-Jordan Elimination}
\begin{definition}
    Let $(A| b)$ be an augmented matrix. You can apply three types of \textbf{row operations} to find the solution to such a system.
    \begin{enumerate}
        \item \textbf{Row Exchange}: Interchange two rows.
        \item \textbf{Scaling}: Multiply a row by a non-zero number.
        \item \textbf{Row Replacement}: Replace row $k$ by its sum with a constant multiple of row $j$. That is, $R_k + \ga R_j$ such that $\ga \neq 0$. 
    \end{enumerate}
\end{definition}

\begin{thm}
    Row operations do not change the solutions to a linear system.
\end{thm}

\begin{prop}
    Each row operation can be realized by multiplication of an invertible matrix. 
    \begin{enumerate}
        \item \textbf{Row Exchange}: Suppose you are changing the $k$th and $j$th rows. This can be expressed as the identity matrix with the  
        $k$th and $j$th rows swapped.
        % \begin{align*}
        %     \begin{pmatrix}
        %         \cdots & I_1 & \cdots \\
        %         & \vdots & \\
        %         \cdots & I_j & \vdots \\
        %         & \vdots & \\
        %         \cdots & I_k & \vdots \\
        %         & \vdots &
        %     \end{pmatrix}
        % \end{align*}
        
        \item \textbf{Scaling}: Suppose you are scaling the $k$th row by $\ga \in \F$. You can represent this as the identity matrix where the
        $k$th row is scaled by $\ga$.        
    \end{enumerate}
\end{prop}

\subsection{Finding the Inverse of a Function via Row Reduction}
\begin{strat}
    Suppose you have a $n \times n$ matrix,
    \begin{align*}
        A &= 
        \begin{pmatrix}
            a_{1,1} & \cdots & a_{1, n} \\
            \vdots & \ddots & \vdots \\
            a_{n, 1} & \cdots & a_{n, n}
        \end{pmatrix}
    \end{align*}

    You can create an augmented matrix, $(A | b)$, where $b$ is the identity matrix.
    \begin{align*}
        \left(\begin{array}{ccc|cccc}
            a_{1,1} & \cdots & a_{1,n} & 1 & 0 & \cdots & 0 \\
            a_{2, 1} & \cdots & a_{2, n} & 0 & 1 & \cdots & 0 \\ 
            \vdots  & \ddots & \vdots  & \vdots & \vdots & \ddots & \vdots \\
            a_{n,1} & \cdots & a_{n,n} & 0 & 0 & \cdots & 0
        \end{array}\right)
    \end{align*}

    Now, make $A$ into its row reduced echelon form. The right side (what was originally the identity matrix) will be transformed into $A$'s inverse
    matrix.
\end{strat}

\subsection{Echelon Form}
\begin{definition}
    A matrix or augmented matrix is in \textbf{echelon form} if it satisfies two conditions.
    \begin{enumerate}
        \item All zero rows are below all non-zero rows.
        \item For a non-zero row, its leftmost non-zero entry (which is called the leading entry or \textbf{pivot}) is strictly to the right of
        the leading entry in the row before it. 
    \end{enumerate}
    \begin{example}
        The following matrix is in echelon form.
        \begin{align*}
            \left(\begin{array}{ccc|c}
                1 & 2 & 0 & 8 \\
                0 & 1 & 1 & 1 \\
                0 & 0 & 3 & 3
            \end{array}\right)
        \end{align*}
    \end{example}
\end{definition}

\begin{thm}
    Every matrix can be transformed into echelon form by applying a finite sequence of row operations.
\end{thm}

\begin{strat} \textbf{(Echelon Form)}

    \textbf{Main Step}:
    \begin{enumerate}
        \item Find leftmost non-zero column matrix.
        \item Apply row-exchange to ensure entry in the first row of the column is non-zero.
        \item Apply scaling so the pivot in the first row equals 1.
        \item Apply row replacement by adding an appropriate multiple of row 1 to each other row, so all entries below the pivot = 0.
    \end{enumerate}

    \textbf{Next}:
    \begin{enumerate}
        \item Leave the first row alone.
        \item Apply the main step to columns 2 through m.
        \item Apply the main step to columns 3 through m.
        \item Continue applying the main step to each subsequent column.
    \end{enumerate}

    Then, you have echelon form.
\end{strat}

\subsection{Row Reduced Echelon Form}
\begin{definition}
    A matrix or augmented matrix is in \textbf{row reduced echelon form} if it fulfills the requirements of echelon form and two more constrains.
    In total, these requirements are summarized below.
    \begin{enumerate}
        \item All zero rows are below all non-zero rows.
        \item The left most non-zero entry in each column is a pivot and every pivot is strictly to the right of every pivot above it.
        \item Every pivot is equal to 1.
        \item All entries above a pivot are zero.
    \end{enumerate}
\end{definition}

\begin{thm}
    Every matrix can be put into reduced echelon form by a finite sequence of row operations.
\end{thm}

\begin{definition}
    The equation $Ax = b$ is \textbf{inconsistent} if it does not have a solution.
\end{definition}

\begin{thm}
    The system $Ax = b$ is inconsistent if and only if there exists a row of the form,
    \begin{align*}
        \begin{pmatrix}
            0 & 0 & \cdots & 0 & | & b
        \end{pmatrix}
    \end{align*}
    Such that $b \neq 0$.
\end{thm}

\begin{corollary}
    A system $Ax = b$ has a solution if there exists a pivot in every row.
\end{corollary}

\begin{corollary}
    A system $Ax = b$ with a solution has a unique solution for $b$ if and only if there exists a pivot in every column of the reduced echelon
    form of $A$.
\end{corollary}

\begin{thm}
    Let $v_1, ..., v_m \in \F^n$ be a system of vectors. Let,
    \begin{align*}
        A &= 
        \begin{pmatrix}
            | & | &  & | \\
            v_1 & v_2 & \cdots & v_m \\
            | & | &  & |           
        \end{pmatrix}
    \end{align*}
    be the $m \times n$ matrix. The system is
    \begin{enumerate}
        \item \textbf{Linearly Independent} if and only if $A_e$ has a pivot in every column (no free variables).
        \item \textbf{Spanning} if and only if $A_e$ has a pivot in every row (solutions exist) .
    \end{enumerate}
\end{thm}