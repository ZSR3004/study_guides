\section{Selected Homework Problems}
This section will include homework problems of importance. That is, homework problems that we might be asked to prove on an exam or problems that might
be fundamental to proving other, commonly talked about problems. I won't be including the proofs here, but if the proof is somehow important because of
the technique used or I thought it was difficult, it will be in the selected proofs section. Note that some of these problems might have already been
mentioned in previous parts of this study guide. I'm just including most everything here for completeness' sake.

\subsection{Homework 3}
\begin{enumerate}
    \item[] \textbf{Problem 2:} 
        \begin{lemma}
        Let $V$ and $W$ be vector spaces. If $T : V \rar W$ is a linear map, then $T(\zV_V) = \zV_W$.
        \end{lemma}
    \item[] \textbf{Problem 4a:}
        \begin{lemma}
            Let $V$ and $W$ be vector spaces. Suppose $v_1, ..., v_k$ is a set of vectors in $V$. If $Tv_1, ..., Tv_k$ is a linearly independent list in
            $W$, then $v_1, ..., v_k$ is a linearly independent list in $V$.
        \end{lemma}
        \begin{remark}
            The converse is not true!
        \end{remark}
\end{enumerate}

\subsection{Homework 4}
\begin{enumerate}
    \item[] \textbf{Problem 1}
    \begin{lemma}
        Suppose $A : V \rar W$ is an isomorphism. If $v_1, ..., v_k$ is a basis for $V$, then $Av_1, ..., Av_n$ is a basis for $W$.
    \end{lemma}
    \item[] \textbf{Problem 2}
    \begin{lemma}
        Suppose $V$ and $W$ are vector spaces over $\F$. Suppose the set $v_1, ..., v_n$ is a basis for $V$ and the set $w_1, ..., w_n$ is a basis
        for $W$. Let $A : V \rar W$ be a linear map defined by $A(v_i) = w_i$ for $i \in \{1, ..., n\}$.

        If $V$ is a vector space with a basis of $n$ elements, then $V$ is isomorphic to $\F^n$.
    \end{lemma}

    The following corollary is not from a homework, but is a natural extension of this lemma using a definition from later in the course.
    \begin{corollary}
        If $\dimV V = n$, then $V$ is isomorphic to $\F^n$.
    \end{corollary} 

    \item[] \textbf{Problem 3a and 3b} 
    \begin{lemma}
        Let $A : V \rar W$ be a linear transformation. The null space of $A$ is s subspace of $V$.
    \end{lemma}
    \begin{lemma}
        Let $A : V \rar W$ be a linear transformation. The range of $A$ is a subspace of $W$.
    \end{lemma}

\end{enumerate}

\subsection{Homework 5}
\begin{enumerate}
    \item[] \textbf{Problem 1}
    \begin{lemma}
        Let $X$ and $Y$ be vector spaces. Let $A : X \rar Y$ be a linear transformation.

        $A$ is invertible \textbf{if and only if} for all $b \in Y$, the equation $Ax = b$ has a unique solution for all $x \in X$.
    \end{lemma}
\end{enumerate}

\subsection{Homework 6}
\begin{enumerate}
    \item[] \textbf{Problem 1}
    \begin{lemma}
        Suppose $V$ is a vector space and $v_1, ..., v_n$ is a \textbf{linearly independent} system in $V$.
        
        $v_1, ..., v_n$ is a basis \textbf{if and only if} $n = \dimV V$.
    \end{lemma}
    \item[] \textbf{Problem 2}
    \begin{lemma}
        Suppose $V$ is a vector space of dimension $n$. A system $v_1, ..., v_n$ in $V$ is linearly independent \textbf{if and only if} it spans $V$.
    \end{lemma}
    \item[] \textbf{Problem 4}
    \begin{thm}
        Let $V$ be a subspace of a vector space $W$ with $\dimV W < \infty$. Then,
        \begin{enumerate}
            \item $V$ is finite dimensional.
            \item $\dimV V \leq \dimV W$, and
            \item if $\dimV V = \dimV W$, then $V = W$.
        \end{enumerate}
    \end{thm}
\end{enumerate}

\subsection{Homework 7}
\begin{enumerate}
    \item[] \textbf{Problem 2}
    \begin{lemma}
        Let $A$ be a matrix. The pivot columns in $A$ form a basis for $\ranV A$.
    \end{lemma}
    \begin{remark}
        You can substitute $A$ for $A_{re}$ and this is still true (since $A_{re}$ is just another matrix). Talking in terms of $A_{re}$ is exactly
        problem 1 of homework 7. In a proof, it might be easier to prove something about $A_{re}$ first, then talk about the orginal matix.
    \end{remark}
    \item[] \textbf{Problem 7}
    \begin{thm} \textbf{(General Rank's Theorem.)}

        Suppose $V$ and $W$ are finite-dimensional vector spaces. Let $A : V \rar W$ be a linear map. Then,
        \begin{align*}
            \dimV V &= \dim (\nullV A) + \dim (\ranV{A}).
        \end{align*}
    \end{thm}
    \item[] \textbf{Problem 8}
    \begin{lemma}
        If $A : X \rar Y$ is a linear map and $V$ is a subspace of $X$, then $\dim (AV) \leq \dimV V$.
    \end{lemma}
\end{enumerate}

\subsection{Homework 8}
\begin{enumerate}
    \item[] \textbf{Problem 1}
    \begin{lemma}
        Let $T: V \rar W$ be a linear transformation. Let $\cA = \{v_1, ..., v_n\}$ be a basis for $V$. Let $\cB = \{ w_1, ..., w_m \}$ be a basis for
        $W$. Finally, let $[T]_{\cA \cB}$ be the matrix of linear transformation $T$ with respect to $\cA$ and $\cB$.
        If $v \in V$, then 
        \begin{align*}
            [T]_{\cA \cB} [v]_{\cA} &= [T(v)]_{\cB}
        \end{align*}
    \end{lemma}
    \item[] \textbf{Problem 2}
    \begin{lemma}
        Let $\cA$ and $\cB$ be bases for vector spaces $V$. Let $[I]_{\cB \cA}$ be the change in coordinates matrix from basis $\cA$ to basis $\cB$.
        \begin{align*}
            [I]_{\cA \cB} &= ([I]_{\cB \cA})^{-1}
        \end{align*}
    \end{lemma}
\end{enumerate}

\subsection{Homework 9}
\begin{enumerate}
    \item[] \textbf{Problem 1b}
    \begin{lemma}
        Suppose matrices $A$ and $B$ are similar matrices. Then,
        \begin{align*}
            \detV A &= \detV B.
        \end{align*}
    \end{lemma}
    \item[] \textbf{Problem 2}
    \begin{lemma}
        Let $A \in \mathbb{M}^{\C}_{n \times n}$. The determinant of $A$ is the product of its eigenvalues. That is, if $\{ \gl_1 , ..., \gl_n \}$ are its
        eigenvalues, then
        \begin{align*}
            \detV A &= \gl_1 \gl_2 \cdots \gl_n.
        \end{align*}
    \end{lemma}

    There was a hint for this problem, which can be expressed as the following lemma.
    \begin{lemma}
        Let $A \in \mathbb{M}^{\C}_{n \times n}$. Then,
        \begin{align*}
            \detV (A - \gl I) &= (\gl_1 - \gl) (\gl_2 - \gl) \cdots (\gl_n - \gl),
        \end{align*}
        Where $\gl_1, \gl_2, ..., \gl_n$ are the eigenvalues of $A$, listed with multiplicity.
    \end{lemma}

    \item[] \textbf{Problem 3a and 3b}
    \begin{lemma}
        Let $A: V \rar V$ be a linear map. Let $U$ be a subspace of $V$. If $U \subset \nullV A$, then $U$ is invariant under $A$.
    \end{lemma}
    \begin{lemma}
        Let $A: V \rar V$ be a linear map. Let $U$ be a subspace of $V$. If $\ranV A \subset U$, then $U$ is invariant under $A$.
    \end{lemma}
\end{enumerate}