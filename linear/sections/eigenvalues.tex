\section{Eigenbases and Eigenspaces}

\subsection{An Introduction to Spectral Theory}

\begin{definition}
    A scalar $\gl \in \F$ is a \textbf{eigenvalue} of a linear map $A: V \rar V$ if there exists a nonzero vector such that 
    \begin{align*}
        Ax &= \gl x
    \end{align*} 
\end{definition}

\begin{definition}
    A vector $v \in V$ is an \textbf{eigenvector} of a corresponding eigenvalue $\gl$ if
    \begin{align*}
        Av &= \gl v
    \end{align*} 
\end{definition}

\begin{definition}
    The set of all eigenvalues of a linear map $A: V \rar V$ is called the \textbf{spectrum} of $A$.
\end{definition}

\begin{definition}
    The set of all eigenvectors associated to an eigenvalue $\gl$ together with the $\{ \zV \}$ is the \textbf{eigenspace} of the linear map $A$
    associated with $\gl$.
\end{definition}

\begin{lemma}
    Let $A : \F^n \rar \F^n$ be a linear map. Let $ \gl \in \F$ be an eigenvalue of $A$. Then,
    \begin{align*}
        \text{$x$ is an eigenvector associated to $\gl$} &\iff x \in \nullV{A - \gl I}
    \end{align*}
\end{lemma}

\begin{corollary}
    \begin{align*}
        (A - \gl I)x &= \zV \\
        & \Updownarrow  \\
        Ax - \gl I x &= \zV \\
        & \Updownarrow  \\
        Ax - \gl x &= \zV \\
        & \Updownarrow  \\
        Ax &= \gl x
    \end{align*}
\end{corollary}

\begin{thm}
    A scalar $\gl$ is an eigenvalue of $A : \F^n \rar \F^n$ $\iff \detV(A - \gl I) = 0$
\end{thm}

\begin{definition}
    The polynomial (with variable $\gl$), $\detV(A \gl I)$ is called the \textbf{characteristic polynomial} of $A$.
    By the theorem, to find the eigenvalues, compute the characteristic polynomial and find its roots.
\end{definition}

\begin{thm} \textbf{(The Fundamental Theorem of Algebra)}
    Any polynomial can be factored across complex roots.
\end{thm}

\subsection{Working in Arbitrary Bases}

\begin{setup}
    Let $A : V \rar V$ be a linear transformation. Suppose $\dimV V = n$. Let $\cB = \{ v_1, ..., v_n \}$ be a basis for $V$.
    \begin{enumerate}
        \item 
        \begin{align*}
            [A]_{\cB \cB} &= 
            \begin{pmatrix}
                | & & | \\
                [A(v_1)]_{\cB} & \cdots & [A(v_n)]_{\cB} \\
                | & & |
            \end{pmatrix}
        \end{align*}
        \item Find the eigenvalues of $[A]_{\cB \cB}$.
    \end{enumerate}

    \emph{We will show that $\detV ([A]_{\cB} - \gl I) = 0$ is independent of the basis.}
\end{setup}

\begin{definition}
    Two square $n \times n$ matrices $A$ and $B$ are \textbf{similar} or \textbf{conjugate} if there exists an invertible matrix $Q$ such that,
    \begin{align*}
        A &= S Q S^{-1}
    \end{align*}
\end{definition}

\begin{lemma}
    If $\cB$ and $\cC$ are 2 bases for $V$ and $A : V \rar V$ is a linear map, then $[A]_{\cB \cB}$ and $[A]_{\cC \cC}$ are similar.
\end{lemma}

\begin{lemma}
    If $A$ and $B$ are similar, they have the same determinant. That is,
    \begin{align*}
        \detV A &= \detV B
    \end{align*}
\end{lemma}

\begin{prop}
    The characteristic polynomials of $[A]_{\cB \cB}$ and $[A]_{\cC \cC}$ are equal.
\end{prop}
\begin{proof}
    Let $R = [A]_{\cB \cB}$ and $T = [A]_{\cC \cC}$. Since these are similar matrices, there exists an invertible matrix $Q$ such that,
    \begin{align*}
        R &= S T S^{-1}.
    \end{align*}
    To show $\detV (T - \gl I) = \detV (R - \gl I)$, it is to show that $R - \gl I$ and $T - \gl I$ are similar matrices (since similar matrices have
    the same determinant). It follows that,
    \begin{align*}
         R - \gl I &= S T S^{-1} - \gl I \\
         &= S T S ^{-1} - \gl S I S^{-1} \\
         &= S (T S^{-1} - \gl I S^{-1}) \\
         &= S (T - \gl I) S^{-1}
    \end{align*}
    So, $R - \gl I$ and $T - \gl I$ are similar matrices as desired.
\end{proof}

\begin{remark}
    Consequently, the spectrum $A : V \rar V$ can be computed with respect to any basis.
\end{remark}

\subsection{Special Cases}
\begin{lemma}
    The eigenvalues of a triangular matrix are its diagonal entries.
\end{lemma}

\begin{corollary}
    The eigenvalues of a diagonal matrix are its diagonal entries.
\end{corollary}

\begin{lemma}
    Let $A$ be an $n \times n$ matrix. Let $\gl_1, ..., \gl_n$ be its complex eigenvalues, listed with multiplicity. Then,
    \begin{align*}
        \detV A &= \gl_1 \gl_2 \cdots \gl_n
    \end{align*}
\end{lemma}

The following definitions were not covered in class, but were used heavily on homework $9$.
\begin{definition}
    Suppose $A$ is an $n \times n$ matrix. $A$ is a \textbf{nilpotent matrix} if there exists a \emph{positive integer} $k$ such that,
    \begin{align*}
        A^k &= 0
    \end{align*}
    Note that this is \emph{not} for all $k$. Only a specific $k$ and all the positive integers above that.
\end{definition}

\begin{definition}
    Suppose $A : V \rar V$ is a linear map. Let $U$ be a subspace of $V$. We call $U$ an invariant under $A$ if for any arbitrary $u \in U$, 
    $Au \in U$. More concisely,
    \begin{align*}
        \forall u \in U, \; Au \in U.
    \end{align*}
\end{definition}

\noindent The following lemma was not explicitly done in class or on homework, but is useful to know.
\begin{lemma}
    If $A$ is an $n \times n$ matrix, it has $n$ eigenvalues by the fundamental theorem of algebra.
\end{lemma} 

