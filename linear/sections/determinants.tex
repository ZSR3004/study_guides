\section{Determinants}

\subsection{Constructing the Determinant}
\begin{definition}
    The \textbf{determinant} of the $2 \times 2$ matrix $A = \vect{a & b \\ c & d}$ is given by
    \begin{align*}
        det A = det\vect{a & b \\ c & d} = \left| \begin{array}{cc} a & b \\ c & d \end{array} \right| = ad - bc \in \F
    \end{align*}
\end{definition}

\begin{remark}
    If the $2 \times 2$ matrix $A = \vect{v_1 v_2}$, then the determinant of $A$ equals the signed area of a parallelogram spanned by the vector columns.
    The sign is positive if the angle from $v_1$ to $v_2$ is less than $\pi$.

    If $A$ has a dimension higher than $2 \times 2$, then the determinant is the signed volume of the parallepiped spanned by the columns vectors of 
    $A$.
\end{remark}


\subsection{Expanding Along Columns and Rows}
\begin{notation}
    The \textbf{$(i,j)$-minor} of $A$, denoted $A_{(i,j)}$ is the $(n-1) \times (n - 1)$ matrix obtained from deleting the $i$th row and $j$th column.
\end{notation}

\begin{definition}
    Let $A$ be an $n \times n$ matrix. The number 
    \begin{align*}
        C_{(i,j)} &= (-1)^{i + j} A_{(i,j)}
    \end{align*}
    Is called the $(i,j)-$cofactor of $A$.
\end{definition}

\begin{definition}
    Suppose $A = (a_{(j, k)})$ is an $n \times n$ matrix. We can compute the determinant of $A$ via a process called \textbf{Cofactor Expansion}.
    We can either \textbf{expand along the $j$th row} or \textbf{expand along the $k$th column}
    \begin{enumerate}
        \item[] \textbf{Expanding Along the $j$th Row:}
        \begin{align*}
            \detV A &= \sum_{k = 1}^{n} a_{j,k} C_{(j,k)} \\
            &= \sum_{k = 1}^{n} a_{j,k} (-1)^{j + k} \detV (A_{(j,k)}) 
        \end{align*}
        Note that you're expanding along $j$, so $k$ is in the sum notation. That is, $j$ remains fixed (for example, you might be expanding along the
        first row, so $j = 1$).
        \item[] \textbf{Expanding Along the $k$th Column:}
        \begin{align*}
            \detV A &= \sum_{j = 1}^{n} a_{j,k} C_{(j,k)} \\
            &= \sum_{j = 1}^{n} a_{j,k} (-1)^{j + k} \detV (A_{(j,k)}) 
        \end{align*}
        Note that you're expanding along $k$, so $j$ is in the sum notation. That is, $k$ is a fixed value.
    \end{enumerate}
\end{definition}

\begin{thm}
    Let $A$ be an $n \times n$ matrix. It doesn't matter if you do cofactor expansion along a row or column (or even which row or column). The result
    will always be the same. That is, you can do cofactor expansion along any row and column to find the determinant of a matrix.
\end{thm}

\begin{remark}
    In practice, its often easier to expand along the row or column with the most amount of $0$ entries. This will allow you to cancel out terms early
    on. Expanding along other rows will yield the same result, but you will have computed more determinants than you would have otherwise had to. 
\end{remark}

\subsection{Special Cases of the Determinant}
\begin{lemma}
    If $A$ is an $n \times n$ matrix with a zero row or zero column, then $\detV A = 0$.
\end{lemma}

\begin{definition}
    If $A$ is an $n \times n$ matrix, we call it \textbf{diagonal} if $a_{j,k} = 0$ for all $a_{j,k}$ where $j \neq k$.
\end{definition}
\begin{definition}
    Suppose $A$ is an $n \times n$ matrix. 

    $A$ is \textbf{upper triangular} if $a_{j,k} = 0$ for all $a_{j,k}$ where $k < j$.
    
    $A$ is \textbf{lower triangular} if $a_{j,k} = 0$ for all $a_{j,k}$ where $k > j$.

    $A$ is a \textbf{triangular matrix} if it is upper triangular or lower triangular.
\end{definition}

\begin{remark}
    A diagonal matrix is a type of triangular matrix. So, any theorem that applies to triangular matrices applies to diagonal ones. Although, not
    all theorem related to diagonal matrices apply to triangular ones.
\end{remark}

\begin{thm}
    If $A$ is a triangular matrix, then $\detV A$ is the produce of the entries in $A$'s main diagonal. So,
    \begin{align*}
        \detV A &= a_{1,1} \times a_{2,2} \times \cdots \times a_{n, n} 
    \end{align*}
\end{thm}

\begin{thm}
    Suppose $A$ is an $n \times n$ matrix. Let $A'$ be the result of switching any two rows or columns of $A$. Then,
    \begin{align*}
        \detV A' &= - \detV A
    \end{align*} 
\end{thm}

\subsection{Determinants and Invertibility}

\begin{setup}
    Suppose $A$ is an $n \times n$ matrix. We will work towards showing that if and only if $\detV A \neq 0$.
    
    What we know:
    \begin{enumerate}
        \item $A$ is invertible if and only if its reduced row echelon, $A_{re}$ form has a pivot in every row and column.
        \item $A_{re}$ is a triangular matrix, so its determinant is the product of its diagonal.
    \end{enumerate}

    From these facts, we can deduced that $A$ is invertible if and only if $\detV A_{re} = 1$. Now, we will give the rest of the building blocks.
\end{setup}

\begin{lemma}
    Suppose $A$ and $B$ are $n \times n$ matrices.
    \begin{align*}
        \detV AB &= \detV A \times \detV B
    \end{align*}
\end{lemma}

We will now address the determinants of each elementary matrix.
\begin{lemma}
    Let $A$ be an $n \times n$ matrix.
    \begin{enumerate}
        \item[] \textbf{Row Exchange:} Suppose $E_{j, k}$ is obtained from swapping the $j$th and $k$th row of $I$. Then,
        \begin{align*}
            \detV E_{j,k} &= -1
        \end{align*}
        \item[] \textbf{Row Scaling:} Suppose $E$ is obtained by scaling the $j$th row of $I$ by $c$. Note that by our construction on row scaling,
        $c \neq 0$.
        \begin{align*}
            \detV E &= c \neq 0
        \end{align*}
        \item[] \textbf{Row Replacement:} Suppose $E$ is obtained from replacing row $k$ with its sum with $c \times$row $j$. Then,
        \begin{align*}
            \detV E &= 1
        \end{align*}
    \end{enumerate}
\end{lemma} 

\begin{corollary}
    The determinants of the elementary matrices are not zero.
\end{corollary}

So, it follows that,
\begin{thm}
    If $A$ is an $n \times n$ matrix, then $A$ is invertible if and only if $\detV A \neq 0$.
\end{thm}

From this, we find three corollaries.
\begin{corollary}
    If $A$ has $2$ identical columns, then $A$ is not invertible.
\end{corollary}
\begin{corollary}
    If $A$ has $2$ identical rows, then $A$ is not invertible.
\end{corollary}

\begin{corollary}
    Let $A$ be invertible, then
    \begin{align*}
        \detV({A^{-1}}) &= \frac{1}{\detV A}
    \end{align*}
\end{corollary}
