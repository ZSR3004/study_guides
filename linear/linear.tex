\documentclass{article}

\usepackage{amssymb}
\usepackage[english]{babel}
\usepackage{amsmath}
\usepackage[margin=1in]{geometry}
\usepackage{algorithm}
\usepackage{algpseudocode}
\usepackage{amsthm}
\usepackage{amsfonts}
\usepackage{graphicx, overpic}
\usepackage{color}
\usepackage{enumerate}
\usepackage{multicol}
\usepackage{mathrsfs}
\usepackage{mathabx}
\usepackage[T1]{fontenc}
\usepackage{tabularx}
\usepackage{siunitx}
\usepackage{hyperref}

\graphicspath{{./images/}}

% Theorem Environments
\theoremstyle{plain}
\newtheorem{thm}{Theorem}[section]
\newtheorem{lemma}[thm]{Lemma}
\newtheorem{sublemma}[thm]{Sub-Lemma}
\newtheorem{prop}[thm]{Proposition}
\newtheorem{corollary}[thm]{Corollary}

% Definition Environments
\theoremstyle{definition}
\newtheorem*{definition}{Definition}
\newtheorem*{example}{Example}
\newtheorem*{remark}{Remark}
\newtheorem*{examples}{Examples}
\newtheorem*{obs}{Observation}
\newtheorem*{claim}{Claim}
\newtheorem*{assumption}{Assumption}
\newtheorem*{notation}{Notation}
\newtheorem*{counter}{Counterexample}

% Algorithms
\newcommand{\quickalg}[1]{\begin{verbatim} #1 \end{verbatim}}
\newcommand{\alg}[2]{
    \begin{algorithm} \caption{#1} \begin{algorithmic}[1]
        #2
    \end{algorithmic} \end{algorithm}}

% Mathbb
\newcommand{\Z}{\mathbb{Z}}
\newcommand{\R}{\mathbb{R}}
\newcommand{\N}{\mathbb{N}}
\newcommand{\Q}{\mathbb{Q}}
\newcommand{\F}{\mathbb{F}}
\newcommand{\C}{\mathbb{C}}

% Mathcal
\newcommand{\cA}{\mathcal{A}}
\newcommand{\cB}{\mathcal{B}}
\newcommand{\cC}{\mathcal{C}}
\newcommand{\cD}{\mathcal{D}}
\newcommand{\cE}{\mathcal{E}}
\newcommand{\cF}{\mathcal{F}}
\newcommand{\cG}{\mathcal{G}}
\newcommand{\cH}{\mathcal{H}}
\newcommand{\cI}{\mathcal{I}}
\newcommand{\cJ}{\mathcal{J}}
\newcommand{\cK}{\mathcal{K}}
\newcommand{\cL}{\mathcal{L}}
\newcommand{\cM}{\mathcal{M}}
\newcommand{\cN}{\mathcal{N}}
\newcommand{\cO}{\mathcal{O}}
\newcommand{\cP}{\mathcal{P}}
\newcommand{\cQ}{\mathcal{Q}}
\newcommand{\cR}{\mathcal{R}}
\newcommand{\cS}{\mathcal{S}}
\newcommand{\cT}{\mathcal{T}}
\newcommand{\cU}{\mathcal{U}}
\newcommand{\cV}{\mathcal{V}}
\newcommand{\cW}{\mathcal{W}}
\newcommand{\cX}{\mathcal{X}}
\newcommand{\cY}{\mathcal{Y}}
\newcommand{\cZ}{\mathcal{Z}}

% Greek
\newcommand{\gt}{\theta}
\newcommand{\gT}{\Theta}
\newcommand{\gl}{\lambda}
\newcommand{\ga}{\alpha}
\newcommand{\gb}{\beta}

% Brackets
\newcommand{\la}{\langle}
\newcommand{\ra}{\rangle}
\newcommand{\lb}{\{}
\newcommand{\rb}{\}}
\newcommand{\lp}{\left(}
\newcommand{\rp}{\right)}

% Signs
\newcommand{\lar}{\leftarrow}
\newcommand{\rar}{\rightarrow}
\newcommand{\lrar}{\leftrightarrow}
\newcommand{\zV}{\mathbf{0}}
\newcommand{\vect}[1]{\begin{pmatrix} #1 \end{pmatrix}}

\title{Linear Algebra Study Guide}
\author{Ziyad Rahman}
\date{\today}

\begin{document}
\maketitle

\tableofcontents
\newpage

\section{An Introduction to Vector Spaces}

\subsection{The Field}
Before we begin discussing what the field is, we need to makes sure we understand what complex numbers are.
\begin{definition}
    \textbf{The complex numbers} is defined as follows.
    \begin{align}
        \C &= \{ a + bi \; | \; a,b \in \R \}
    \end{align}
\end{definition}

\begin{definition}
    \textbf{A Field} refers to the space over which a vector space is defined. In this class we will often refer to a special field (which we may refer to
    simply as "the field). It is defined as follows.
    \begin{align}
        \F &= \{ \R \cup \C \}
    \end{align}
    That is, the vector space is either defined over the real or complex numbers.
\end{definition}

\begin{remark}
    If $\F = \R$, then $V$ is called a \underline{real vector space}. \\
    If $\F = \C$, then $V$ is called a \underline{complex vector space}.
\end{remark}

\subsection{Defining Vector Spaces}
\begin{definition}
    A \textbf{vector space} $V$ over $\F$ is a set of elements called \textbf{vectors} that have two operations, closed addition and closed scalar 
    multiplication. These operations satisfy 8 axioms. That is,
    \begin{align*}
        \text{Addition: } & f : V \times V \rar V \\
        \text{Scalar Multiplication: } & g : \F \times V \rar V
    \end{align*}
\end{definition}

\begin{thm}
    If $V$ is a vector space, then it must have closed addition and closed scalar multiplication such that they fulfill the following vector space 
    axioms.
    \begin{enumerate}
        \item \textbf{Commutativity:} $v + w = w + v$ for all $v,w \in V$.
        \item \textbf{Associativity:} $(v + w) + u = v + (u + w)$ for all $v,w,u \in V$.
        \item \textbf{Additive Identity:} There exists (a unique) $\zV \in V$ such that $v + \zV = v$ for all $v \in V$.
        \item \textbf{Additive Inverse:} For all $v \in V$, there exists (a unique) $v' \in V$ such that $v + v' = \zV$.
        \item \textbf{Multiplicative Identity:} $1 \times v = v$ for all $v \in V$ and $1 \in \F$.
        \item \textbf{Multiplicative Associativity:} $(\ga \gb) v = \ga (\gb v)$ for all $v\in V$ and $\ga, \gb \in \F$.
        \item \textbf{Scalar Distribution:} $\ga (v + w) = \ga v + \ga w$ for all $v,w \in V$ and $\ga \in \F$.
        \item \textbf{Vector Distribution:} $v (\ga + \gb) = \ga v + \gb v$ for all $v\in V$ and $\ga, \beta \in \F$.
    \end{enumerate}
\end{thm}

\begin{remark} 
    The first 4 axioms relate to addition. The next 2 are about multiplication. The final two connect addition and multiplication through the distributive 
    property.
\end{remark}

\begin{lemma}
    If $V$ is a vector space, then $\zV \times v = \zV$ for all $v \in V$.
\end{lemma}

\subsection{Important Vector Spaces and Vector Space Properties}
This section will basically be a bunch of lemmas and theorems that state "this thing" is a vector space.
\begin{thm}
    $\R^n$ is a vector space over $\R$.
\end{thm}
\begin{thm}
    $\C^n$ is a vector space over $\C$.
\end{thm}

\begin{thm}
    If $M^\F_{m \times n}$ denotes the set of all $m \times n$ matrices with entries in $\F$, then $M^\F_{m \times n}$ is a vector space over $\F$.
\end{thm}

\begin{thm}
    Let $\mathbb{P}^\F_n$ denote the set of all polynomials of degree $\leq n$ and coefficients in $\F$. That is,
    \begin{align*}
        \mathbb{P}^\F_n &= \{ p(t) = a_0 + a_1 t + a_2 t^2 + \cdots + a_n t^n \; | \; a_i \in \F \text{ for } i \in \{ 0, ..., n \} \}
    \end{align*}

    $\mathbb{P}^\F_n$ is a vector space over $\F$.
\end{thm}

\section{Bases}

\subsection{Linear Combinations}
\begin{definition}
    Let $V$ be a vector space over $\F$ and $v_1, ..., v_p \in V$ be a list of vectors. A \textbf{linear combination} of these vectors is a sum of the
    form,
    \begin{align*}
        \ga_1 v_1 + \cdots \ga_p v_p &= \sum_{i = 1}^{p} \ga_i v_i \text{ where $\ga_i \in \F$}
    \end{align*}
\end{definition}

\subsection{Linear Independence}
\begin{definition}
    A linear combination, $\sum_{i = 1}^{p} \ga_i v_i$ is said to be the \textbf{trivial combination (or trivial case)} if $a_i = 0$ for all $i$.
\end{definition}

\begin{definition}
    A system of vectors is \textbf{linearly independent} if only the trivial linear combination of $v_1, ..., v_p$ is equal to the zero vector. That is, 
    if
    \begin{align*}
        \sum_{i = 1}^{p} \ga_i v_i &= \zV
    \end{align*}

    Then, $\ga_1 = \ga_2 = \cdots = \ga_p = 0 \in \F$.
\end{definition}

\begin{definition}
    A system of vectors is \textbf{linearly dependent} if it is not linearly independent. We can represent this in two ways. First, there exists a 
    linear combination of this list for the zero vector such that the linear combination is not the trivial one.
    \begin{align*}
        \sum_{i = 1}^{p} \ga_i v_i &= \zV \text{ but there exists a $\ga_i$ such that $\ga_i \neq 0$}
    \end{align*}

    Second, that a list is linearly dependent if and only if one of the vectors in the list can be written as a linear combination of the other 
    vectors in that list.
    \begin{align*}
        v_k &= \sum_{\substack{i = 1 \\ i \neq k}}^{p} \ga_i v_i \\
        &= \ga_1 v_1 + \cdots + \ga_{k - 1} v_{k - 1} + \ga_{k + 1} v_{k + 1} + \cdots + \ga_p v_p
    \end{align*}

    Technically, this second definition follows from Proposition 2.6 from the textbook, but its so fundamental (and rather easy to prove, since
    it directly follows the first definition) that I've decided to include it here.
\end{definition}

\subsection{Spanning}
\begin{definition}
    A collection of vectors is said to \textbf{span} (or \textbf{generate, be a spanning system of, be a complete system of}) $V$ if any vector
    $v \in V$ can be written as a linear combination of that collection.

    More formally, suppose you have a list $v_1, ..., v_p \in V$ and an arbitrary $v \in V$. This list spans only if
    \begin{align*}
        \sum_{i = 1}^{p} \ga_i v_i &= v
    \end{align*}
\end{definition}

\subsection{Defining Bases}
\begin{definition}
    A system of vectors $v_1, ..., v_p \in V$ is a \textbf{basis} for $V$ if every vector $v \in V$ admits a unique representation as a linear 
    combination of $v_1, ..., v_p$. 
    
    That is, a system of vectors is a basis if that list is linearly independent and spans $V$.
\end{definition}

\begin{remark}
    Let $v \in V$ be arbitrary. If the list $v_1, ..., v_p \in V$ is a basis, then
    \begin{align*}
        \sum_{i = 1}^{p} \ga_i v_i &= v
    \end{align*}

    We would refer to the $\ga_i$ as the \textbf{coordinates} of $v$ with respect to this basis.
\end{remark}

\begin{thm}
    Any finite set of vectors that span a vector space contains a basis.
\end{thm}
\begin{remark}
    The proof is "recursive" in that the idea is you can remove elements of that list until it becomes linearly independent. Removing elements
    doesn't impact the list's ability to span.
\end{remark}

\begin{definition}
    The \textbf{standard basis vectors} refers to a simple, reliable basis for various vector spaces. The two important ones are for $\F^n$ and 
    $\mathbb{P}^\R_n$.

    The standard basis vectors for $\F^n$ are
    \begin{align*}
        e_1 = \vect{1 \\ 0 \\ \vdots \\ 0}, e_1 = \vect{0 \\ 1 \\ \vdots \\ 0}, \cdots, e_n = \vect{0 \\ 0 \\ \vdots \\ 1} \in \F^n
    \end{align*}

    The standard basis vectors for $\mathbb{P}^\R_n$ are
    \begin{align*}
        1, t, t^2, ..., t^n \in \mathbb{P}^\R_n
    \end{align*}

\end{definition}


\section{Linear Transformations}

\subsection{Introducing Linear Transformations}
\begin{definition}
    A \textbf{transformation} $T$ form a set $X$ to $Y$ is a function from $X$ to $Y$. We denote it as
    \begin{align*}
        T : X \rar Y
    \end{align*}
\end{definition}

\begin{definition}
    Suppose there are two vector spaces $V$ and $W$ both over $\F$. A transformation between them is \textbf{linear} if it holds for two properties.
    \begin{enumerate}
        \item \textbf{Additivity:} $T(v + u) = T(v) + T(u)$ for all $v,u \in V$.
        \item \textbf{Homogeneity:} $T(\ga v) = \ga T(v)$ for all $v \in V$ and $\ga \in \F$.
    \end{enumerate}
\end{definition}

\begin{lemma}
    Let $T : \mathbb{P}^\R_n \rar \mathbb{P}^\R_{n - 1}$ be the differentiation of polynomials. $T$ is a linear map.
\end{lemma}

\subsection{Representing Transformations as Matrices}

\subsubsection{Transformations as Matrices}

\begin{definition}
    Suppose
    \begin{align*}
        T (e_k) = \vect{a_{1, k} \\ a_{2, k} \\ \vdots \\ a_{n, k}} \in \F^n
    \end{align*}
    
    Then, the \textbf{matrix representing $T$} is defined as the following.
    \begin{align*}
        A &= 
        \begin{pmatrix}
            a_{1, 1} & a_{1, 2} & \cdots & a_{1, n} \\
            a_{2, 1} & a_{2, 2} & \cdots & a_{2, n} \\
            \vdots & \vdots & \ddots & \vdots \\
            a_{m, 1} & a_{m, 2} & \cdots & a_{m, n}
        \end{pmatrix}
    \end{align*}

    Basically, the first column is just $T(e_1)$. The second column is $T(e_2)$, so the $nth$ column is $T(e_n)$.
\end{definition}

\subsubsection{Matrix-Vector Mutliplication}
The concept of matrix-vector multiplication is the ability to find $T(v)$ through the use of a matrix.

\begin{definition}
    Let $A$ be the matrix representing $T$ and $x \in \F^n$. Then,
    \begin{align*}
        Ax &= T(x)
    \end{align*}

    More explicitly,
    \begin{align*}
        Ax &= 
        \begin{pmatrix}
            a_{1, 1} & a_{1, 2} & \cdots & a_{1, n} \\
            a_{2, 1} & a_{2, 2} & \cdots & a_{2, n} \\
            \vdots & \vdots & \ddots & \vdots \\
            a_{m, 1} & a_{m, 2} & \cdots & a_{m, n}
        \end{pmatrix}
        \vect{x_1 \\ x_2 \\ \vdots \\ x_n}
        \\ &=
        \begin{pmatrix}
            a_{1, 1} x_1 + a_{1, 2} x_2 + \cdots + a_{1, n} x_n \\
            a_{2, 1} x_1 + a_{2, 2} x_2 + \cdots + a_{2, n} x_n \\
            \vdots \\
            a_{m, 1} x_1 + a_{m, 2} x_2 + \cdots + a_{m, n} x_n
        \end{pmatrix}
    \end{align*}

\end{definition}

\subsubsection{Composing Linear Maps}

\subsubsection{Matrix-Matrix Multiplication}
\begin{definition}
    Let $A$ be a $m \times n$ matrix and $B$ be a $n \times r$ matrix. We have two definitions for \textbf{matrix-matrix multiplication}.
    \begin{enumerate}
        \item Suppose,
        \begin{align*}
            B &= 
            \begin{pmatrix}
                | & | &  & | \\
                b_1 & b_2 & \cdots & b_r \\
                | & | &  & | \\
            \end{pmatrix}
        \end{align*}
        Where $b_i$ is the $ith$ column of $B$. Then,
        \begin{align*}
            AB &:= 
            \begin{pmatrix}
                | & | &  & | \\
                Ab_1 & Ab_2 & \cdots & Ab_r \\
                | & | &  & | \\
            \end{pmatrix}
        \end{align*}

        \item The entry $(AB)_{j,k}$ is
        \begin{align*}
            (jth \text{ row of $A$}) \times (ith \text{ column of $B$}) &= \sum_{l = 1}^{n} a_{j,l} \times b_{l,k}
        \end{align*}
    \end{enumerate}
\end{definition}

\begin{remark}
    Matrix multiplication holds for the following properties.
    \begin{enumerate}
        \item \textbf{Associativity:} $A(BC) = (AB)C$
        \item \textbf{Matrix Distribution:} $A(B + C) = AB + AC$ and $(A + B)C = AC + BC$
        \item \textbf{Scalar Distribution:} $A(\ga B) = \ga (AB)$
    \end{enumerate}

    Very importantly, \textbf{matrix multiplication is NOT commutative.}
\end{remark}

\begin{thm}
    Let $A$ be the matrix representing $T_1$ and $B$ be the matrix representing $T_2$. Then, the matrix $BA$ represents the composition $T_2 \circ T_1$.
\end{thm}

\subsubsection{The Rotational Matrix} % NEED MORE INFO
\begin{definition}
    Let the \textbf{rotational matrix} be defined as follows.
    \begin{align*}
        R_\gt &=
        \begin{pmatrix}
            \cos \gt & - \sin \gt \\
            \sin \gt & \cos \gt
        \end{pmatrix}
    \end{align*}

    When multiplied by a vector, it rotates the vectors $\gt$ clockwise. You are functionally composing a vector with a rotation.
\end{definition}

\subsubsection{Transposing and Tracing Matricies}
Two more useful tools to have are as follows.
\begin{definition}
    The \textbf{transpose}of $A$ is a matrix 
    \begin{align*}
        A^T &= (a'_{i,j})_{\substack{1 \leq j \leq n \\ 1 \leq i \leq m}} \text{ and } a'_{i,j} = a_{j,i}
    \end{align*}
\end{definition}

\begin{definition}
    The \textbf{trace} of an $n \times n$ matrix $A = (a_{j,k})$ is the sum of the diagonal entries.
    \begin{align*}
        tr(A) &= a_{1, 1} + a_{2, 2} + \cdots + a_{n, n}
    \end{align*} 

    Equivalently,
    \begin{align*}
        tr(A) &= \sum_{i = 1}^{n} a_{i, i}
    \end{align*}
\end{definition}

\subsection{Subspaces}
\begin{definition}
    Suppose $V$ is a vector space. $V_0$ is a \textbf{subspace} of $V$, if the following properties hold.
    \begin{enumerate}
        \item $V_0 \subset V$
        \item $V_0 \neq \varnothing$
        \item $v + w \in V_0$ for all $v,w \in V_0$
        \item $\ga v \in V$ for all $v \in V$ and $\ga \in \F$
    \end{enumerate}
\end{definition}

\subsubsection{Null Spaces and Range}
\begin{definition}
    Suppose $T : V \rar W$. The \textbf{null space or kernel} of $T$ is the set of all the values $v_0 \in V$ such that
    \begin{align*}
        T(v_0) &= \zV_W
    \end{align*}
\end{definition}

\begin{definition}
    Suppose $T : V \rar W$. The \textbf{range} of $T$ is the set of all values $w \in W$ such $w$ is in the image of $T$.
\end{definition}

\subsection{Isomorphisms}
\subsubsection{Inverses}
\begin{definition}
    Suppose $A : T \rar W$ is a linear transformation. $A$ is \textbf{right invertible} if there exists a $B : W \rar V$ such that $B \circ A = I_V$.
    In matrix multiplication, $BA$ is the identity matrix.
\end{definition}

\begin{definition}
    Suppose $A : T \rar W$ is a linear transformation. $A$ is \textbf{left invertible} if there exists a $B : W \rar V$ such that $B \circ A = I_W$.
    In matrix multiplication, $AB$ is the identity matrix.
\end{definition}

\begin{definition}
    A linear transformation $A : V \rar W$ is \textbf{invertible} only if it is right invertible and left invertible.
\end{definition}

From that, we have the following corollary.
\begin{corollary}
    A linear transformation $A : V \rar W$ is invertible if and only if there exists a unique linear transformation $A^{-1}: W \rar V$ such that
    \begin{align*}
        A^{-1} \circ A = I_V \text{ and } A \circ A^{-1} = I_W
    \end{align*}

    We call $A^{-1}$ the inverse of $A$.
\end{corollary}

\begin{lemma}
    If $A$ and $B$ are invertible linear transformations, and $AB$ is defined, then $A \circ B$ is invertible and
    \begin{align*}
        (A \circ B)^{-1} &= B^{-1} \circ A^{-1}
    \end{align*}
\end{lemma}

\begin{thm}
    Let $A : X \rar Y$ be a linear transformation. Then, $A$ is invertible if and only if for all $b \in Y$, the equation
    \begin{align*}
        Ax &= b
    \end{align*}
    has a unique solution.
\end{thm}

\subsubsection{Defining Isomorphisms}
\begin{definition}
    An \textbf{Isomorphism} from a vector space $V$ to a vector space $W$ is an invertible linear map from $V$ to $W$. If there exists an isomorphism
    between two vector spaces, then we would say that those vector spaces are \textbf{isomorphic}.
\end{definition}

We will now list the foundational theorems for isomorphisms.
\begin{thm}
Let $A : V \rar W$ be an isomorphism. If $v_1, ..., v_n$ is a basis for $V$, then $A(v_1), ..., A(v_n)$ is a basis for $W$.
\end{thm}

\begin{thm}
    Let $v_1, ..., v_n$ be a basis for $V$ and $w_1, ..., w_n$ be a basis for $W$. Let $A : V \rar W$ be the map defined by
    \begin{align*}
        A(v_i) &= w_i \text{ for } 1 \leq i \leq n.
    \end{align*}

    $A$ is an isomorphism.
\end{thm}

\begin{corollary}
    If vector spaces have bases of the same size, then they are isomorphic.
\end{corollary}

\end{document}