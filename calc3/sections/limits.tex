\section{Limits}

Limits operate in much of the same way as in single variable calculus. I won't dive into the specific definition since its a little convoluted and
unnecessary to know for our purposes. Know this: in single variable calculus, there are two ways to approach a limit, but in multivariable calculus,
there are an infinite number of ways to approach $(x,y) = (a,b)$. \textbf{For a limit to exist, all approaches must work. If even one of them doesn't
exist, then there is no limit. If you take to limits from different directions and they don't result in the same output, then the limit does not exist.}

There are a few types of functions that we know always have limits.
\begin{enumerate}
    \item Polynomial Functions
    \item ???
\end{enumerate}

So, if you can reduce a function into one of those forms (or rather manipulate it to show that it is one of those functions), then you know 
that function has a limit.