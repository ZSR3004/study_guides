\section{Planes}

\begin{definition}
    A \textbf{plane} in $\R^3$ is uniquely determined by a point in the plane $P_0 = (x_0, y_0, z_0)$ and the normal vector $\vec{n} = \vect{a, b, c}$.
    \begin{align*}
        ax + by + cz &= d
    \end{align*}

    For some constant $d$ such that $d = ax_0 + by_0 + cz_0$. So, the equation for a plane is more explicitly,
    \begin{align*}
        ax + by + cz &= ax_0 + by_0 + cz_0
    \end{align*}

    If you know a second point on the plane $P = (x, y ,z)$, you can derive this formula through the following method. First, find the vector
    between $P_0$ and $P$.
    \begin{align*}
        \vec{P_0 P} &= \vect{x - x_0, y - y_0, z - z_0}
    \end{align*}

    This produces a vector orthonormal to the plane (since its a point on the plane pointing to another point on the plane). Since the normal vector
    was defined to be normal to the plane, we know its also normal to this vector. So, the angle between the $\vec{n}$ and $\vec{P_0 P}$ is 
    $\frac{\pi}{2}$. Therefore,
    \begin{align*}
        \vec{n} \cdot \vec{P_0 P} &= 0 \\
        &= a(x - x_0) + b(y - y_0) + c(z - z_0)
    \end{align*}

    Isolating the constants to one side, this is exactly the equation of a plane.
\end{definition}

\begin{remark}
    Given three points on a plane, $P, Q, R$, you can find the normal vector (and the equation of the plane) by using the cross product.
    \begin{align*}
        \vec{n} &= \vec{P Q} \times \vec{PR}
    \end{align*}
\end{remark}