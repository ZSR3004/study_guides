\section{Torques}
Torques are the rotational equivalent to forces.
If forces cause translational acceleration, motivating an object to move translationally, then torque
motivates objects to move rotationally. Before we talk about torques, we'll first discuss the mass equivalent for rotational motion, then talk about
Newton's Rotational Laws and a few important scenarios.

To law a few ground rules, we represent torques with $\tau$ in $Nm$. The counter-clockwise direction is positive and the clockwise direction is negative.
We can calculate torque with the following equation.
\begin{eqnarray}
    \vec{\tau}  &=& \vec{F} \times \vec{r} \\
                &=& |\vec{F}| |\vec{r}| sin\theta
\end{eqnarray}
Where $\vec{r}$ is how far from the pivot the force is being applied. Notice how the first iteration is vector multiplication, in particular the cross 
product. Given that, we can rewrite this equation in the second way. However, that relies on the fact that the problem is simplified to
fewer dimensions, but since the vast majority of problems we'll deal with are of this simplified form, its worth noting here.

\subsection{Moments of Inertia}
If mass is a measurement of how inert something is that means it measures how difficult something is to move (or otherwise to gain momentum). Likewise,
the \textbf{moment of inertia} measures how difficult it is to rotate something. More intuitively, it measures how far apart mass is spread from
something's pivot (the point it spins around). If two things have the same mass and one is packed closer, then it will have a smaller moment of inertia
compared to the second object who has the same mass spread across a larger distance (for example, if you have two balls with the same mass, the smaller
one has a smaller moment of inertia because the mass is spread out less). We represent the moment of inertia with the variable $I$ and its units are
$kgm^2$.

The moment of inertia depends on the shape of an object, so each one has a different formula (you use a different formula to calculate a disc's moment
of inertia than for a ball). These will always be given, so I won't bother detailing them here.

\subsubsection{The Parallel Axis Theorem}
This theorem is used to find the moment of inertia for an object who's pivot axis has moved but stayed parallel to itself. For example, if spin a ball
on my finger, then its axis is straight up (this would be $I_{original}$). If I instead decide to spin it by spinning in a circle with my arms out, then 
its axis is moved away from the center, but is still parallel (that is, straight up). To calculate the new moment of inertia, we can use the following
equation.
\begin{equation}
    I_{new} = I_{original} + md^2
\end{equation}
Where,
$m$ is the object's mass and \\
$d$ is the distance the axis moved.

\subsection{Newton's Rotational Laws}
Just like with forces, we have three laws to consider. We'll look at them out of order since the second one is more involved (and arguably important)
than the others. The same constraint that this must happen in an inertial reference frame applies.

\subsubsection{Newton's First Rotational Law}
In an inertial reference frame, if there is no torque acting on an object, then objects at rest stay at rest and objects in motion stay in motion with 
a constant velocity.

\subsubsection{Newton's Third Rotational Law}
In an inertial reference frame, torques are met with an equal and opposite reaction.
\begin{equation}
    \tau_{A on B} = - \tau_{B on A}
\end{equation}

\subsubsection{Newton's Second Rotational Law}
This is the same as Newton's Second Translation Law.
\begin{equation}
    \tau_{net} = I \alpha
\end{equation}