\section{Work, Energy, and Power}
\subsection{Energy}
Energy, somewhat recursively, is the ability to do work (more on work
in a little bit). Energy is measured in \textit{joules} ($J$), which is equivalent to
$Nm$.

\subsubsection{Conservation of Energy}
Energy cannot be created or destroyed, only transferred. That means
that at some point A, the energy is equal to the energy at another
point B (if the system encompasses everything).

\begin{eqnarray} \nonumber
    E_A = E_B
\end{eqnarray}

\subsection{Work}
Work (joules) is the measure of energy transferred to an object when it is 
moved over a distance. The actual equation is as follows.

\begin{eqnarray} \label{eq : true work}
    \int \vec{F} \, \text{d}\vec{r}
\end{eqnarray}

\noindent Since this is technically not a calculus based course, we simplify
this equation in the following way.

\begin{eqnarray} \label{eq : work}
    W = \vec{F} \cdot \Delta \vec{r}
\end{eqnarray}

\noindent From this, we can draw a few conclusion that can simplify 
problem-solving. Expanded, we can rewrite the work formula as the
following.

\begin{eqnarray} \nonumber
    \vec{F} \cdot \Delta \vec{r} = F \Delta r \cos(\theta_{\vec{F} \Delta \vec{r}})
\end{eqnarray}

We can take the $\cos(\theta_{\vec{F} \Delta \vec{r}})$ term. From this we
can tell a few things. If the movement and the force are in the same
direction, then the work is positive. If they are in the opposite
direction, then work is negative. Last, if they are perpendicular,
the work is zero.

\subsection{Types of Energy}
There are several types of energy, but there are a few that are especially
important for this course. This section will go through each of these.

\subsubsection{Kinetic Energy}
Kinetic energy refers to the amount of energy in an object by
virtue of its motion.

\begin{equation}
    K = \frac{1}{2}mv^2 = \frac{p^2}{2m}
\end{equation}

We also have the rotational equivalent. This is the only energy type with a rotational equivalent.
\begin{equation}
    K_{Rotational} = \frac{1}{2} I \omega ^2
\end{equation}

\subsubsection{(Gravitational) Potential Energy}
Gravitational potential energy is the amount of stored energy in an
object relative to various parts of its system.

\begin{eqnarray}
    U_g = mg \Delta \vec{r}
\end{eqnarray}

\noindent Where, $\Delta \vec{r}$ is the displacement from an arbitrarily
chosen point. This will typically be a point that is chosen because it makes
calculations easier.

\subsubsection{Elastic (Potential) Energy}
The amount of energy stored in a spring.

\begin{eqnarray}
    U_s = \frac{1}{2} k \Delta \vec{x} \, ^2
\end{eqnarray}
Where, $\Delta \vec{x} \, ^2$ is the displacement of the spring
compared to where it in its equilibrium position.

\subsubsection{Total Energies}
We can write the total potential of a system as
\begin{equation}
    U = U_g + U_s
\end{equation}

We'll quickly relate this back to force with this equation.
\begin{equation}
    F = - \frac{\text{d}U}{\text{d}x}
\end{equation}

For conservative systems, we can also write the following relationships.
\begin{eqnarray}
    \Delta K = - \Delta U \\
    E_{Mechanical} = K + U
\end{eqnarray}

In the real world, energy can be "lost". As a block slows to rest due to friction, some of the block's energy is loss to heat. However, at a higher 
level, the energy is just transferred to the air, so the energy is actually not lost. This is a scenario that is mostly more complicated than what we 
will cover in class. However, we still have tools to figure out how much energy was lost from friction and such. In these types of scenarios its often
helpful to use the \textbf{Work-Energy Theorem}. 
\begin{equation}
    \Delta K = \Delta W_{Conservative} + \Delta W_{Non-Conservative} = \Delta U + \Delta W_{Non-Conservative}
\end{equation}

\subsection{Power}
The last thing we'll talk about is \textbf{power}, which is change in work over time. We'll measure this in $\frac{J}{s}$ or Watts ($W$).
\begin{equation}
    P = \frac{W}{t}.
\end{equation}